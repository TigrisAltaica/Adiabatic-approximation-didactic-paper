\documentclass[12pt]{article}
\usepackage[utf8]{inputenc}
\usepackage{graphicx}
\usepackage{dcolumn}
\usepackage{bm}
\usepackage{braket}
\usepackage{physics}
\usepackage{amsmath}
\usepackage{amssymb}
\usepackage{blkarray}
\usepackage{cancel}

\newcommand{\superket}[1]{|#1\rangle\rangle}
\newcommand{\superbra}[1]{\langle\langle #1 |}
\newcommand{\superbracket}[2]{\langle\langle #1 |#2\rangle\rangle}
\newcommand{\supermatrix}[2]{|#1\rangle\rangle\langle\langle #2|}
\newcommand{\supermel}[3]{\langle\langle #1 |#2|#3 \rangle\rangle}

\begin{document}

\title{Adiabatic approximation didactic paper}
\author{Pablo Enrique Yanes Thomas}
\date{\today}


\maketitle

\section{Introduction}
 
 It is not rare to encounter systems with a wide variety of evolution time-scales or large gaps between energy levels. When modeling these types of systems it is often advantageous or even necessary to approximate the system via an effective model where the slow or low energy parts of the system dominate the dynamics. This type of procedure is called adiabatic elimination \cite{GardinerStochasticMethods1983}. The concept of adiabatic elimination techniques is related to the concept of the adiabatic approximation \cite{HakenSynergetics2013}, where it is assumed that the dynamics of a system with slow and fast time scales are determined by the evolution in the slow time scale.
 

 Adiabatic elimination techniques are widely employed in current research. In one particular example, an ancillary system with continuous output can be employed to measure a quantum system that interacts with the ancillary system. If this interaction is slower than the evolution of the ancillary system, the ancillary degrees of freedom can be adiabatically eliminated yielding a simpler system \cite{CernotikAdiabatic2015}. A different study analyzed an approach to produce an antibunched phonon field in a membrane-in-the-middle optomechanical setup where the cavity field is later adiabatically eliminated \cite{SeokAntibunching2017}. Adiabatic elimination techniques were also employed to eliminate the highly dissipative modes and simplify the dynamics of a hybrid optomechanical system consisting of an atomic ensemble trapped in an optomechanical cavity, with the aim of generating squeezing of the mechanical modes \cite{BaiMirrorSqueezing2019}. In a study analyzing the phase transitions in a hybrid system composed of an ultra cold atomic gas coupled to a mechanical membrane via a light field, adiabatic elimination is used to eliminate the light field and derive an effective Hamiltonian \cite{MannPhaseTransitions2017}. An effective beam splitter Hamiltonian can be used to describe a hybrid system composed of a trapped Bose-Einstein Condensate and an optomechanical cavity if the light field is adiabatically eliminated \cite{SinghStateTransfer2012}. Work has also been done aimed at studying first order corrections to and effective model obtained via adiabatic elimination \cite{JiangAdiabaticCorrection2016}. An effective Hamiltonian for a two qubit gate formed by two atoms in a lambda configuration in an optomechanical cavity can be obtained by adiabatically eliminating lossy channels \cite{ChauhanSwapGate2019}.
 
 Two examples which will be relevant to this paper are the three level atom in a lambda configuration and cavity optomechanical cooling.  In the case of a three level atom in a lambda configuration, the system is usually assumed to have two "low" energy levels that are close together and a "high" energy level that is far away from the other two levels \cite{BrionAdiabatic2006}. In the case of an optomechanical cavity with both mechanical and electromagnetic dissipation for example, the relevant frequencies involved can differ by several orders of magnitude \cite{VezioOMExperiment2020}.

The adiabatic elimination technique that will be presented in this paper relies on projection super-operators. The technique involves splitting the eigenvalue space of the original problem into two spaces, usually a "fast" evolving space and a "slow" evolving space. Once this split has been performed, an equation, or set of equations, that model the evolution of the "slow" space is obtained. The technique will be explained in detail using two different physical examples, a three level atom in a lambda configuration and an optomechanical cavity with dissipation. The concept of super-kets is also introduced in order to simplify the algebra in the case of the tree level atom.
 

\section{Quantum mechanics in Liouville space}

It is important to explain the mathematical spaces that will be involved in the following calculations. The wave function in quantum mechanics is a vector in Hilbert space. This Hilbert space is denoted as $\mathcal{H}_d$, where $d$ is the dimension of the Hilbert space. The system's Hamiltonian is an operator and belongs to a different space, also a Hilbert space, of dimension $d^2$, which is denoted by $\mathcal{O}_d$ as this is the space of Linear Operators, which act on the vectors that are elements of $\mathcal{H}_d$. Elements of these two Hilbert spaces are often represented as vectors and matrices respectively. For the projection technique that will be employed later on, it is useful to work with the Liouville equation, which models the time evolution of the density operator, which is an element of $\mathcal{O}_d$. It will be useful to define a new space of \textit{vectorized} elements of $\mathcal{O}_d$. This space has the same dimension as $\mathcal{O}_d$ and will be denoted by $\mathcal{L}_d$. It is also necessary to define the space of "super operators" which act on elements of $\mathcal{L}_d$ and which will be denoted as $\mathcal{S}_{d^2}$. These spaces, as well as the "super ket" formalism which will be employed are discussed in depth in \cite{GyamfiLiouvilleSpace2020}. The super ket formalism consists of defining the spaces $\mathcal{L}_d$ and $\mathcal{S}_{d^2}$ which allows for quantum mechanics to be done using density operators and the super operators that act on them as if they were vectors and matrices respectively. The elements of $\mathcal{L}_d$ are obtained via the bra flipper operator. Given an element $\mathcal{O}_d$, written as

\begin{equation}
    A = \ketbra{a}{b},
\end{equation} applying the bra flipper operator $\Upsilon$ yields

\begin{equation}
    \Upsilon(A) = \ket{a} \otimes \ket{b}^* =\superket{a,b}.
\end{equation} The symbol $\otimes$ denotes a Kroenecker product. The Kronecker product has the following properties \cite{FernandezKronecker2016}

\begin{align*}
    A \otimes (B+C) =& A \otimes B + A \otimes C\\
    (A+B) \otimes C =& A \otimes C + B \otimes C\\
    (kA)\otimes B =& A\otimes (kB) = k(A \otimes B), \quad k \in \mathbb{C}\\
    (A \otimes B) \otimes C =& A \otimes (B \otimes C).
\end{align*} In matrix representation, the product works as

\begin{equation}
\begin{pmatrix}
A_{11} & A_{12} \\
A_{21} & A_{22}
\end{pmatrix}
\otimes
\begin{pmatrix}
A_{11} & A_{12} \\
A_{21} & A_{22}
\end{pmatrix} =
\begin{pmatrix}
A_{11} B_{11} & A_{11} B{12} & A_{12} B_{11} & A_{12} B_{12} \\
A_{11} B_{21} & A_{11} B{22} & A_{12} B_{21} & A_{12} B_{22} \\
A_{21} B_{11} & A_{21} B{12} & A_{22} B_{11} & A_{22} B_{12} \\
A_{21} B_{21} & A_{21} B{22} & A_{22} B_{21} & A_{22} B_{22}
\end{pmatrix}.
\end{equation}
In this way, density operators can be treated as vectors. Note the rules of the Kroenecker product mean that dimension is retained. In the case of the density operator for a system such as the three level atom, for example, a $3x3$ matrix becomes a $9x1$ vector. All the aforementioned spaces have adjoint or dual spaces associated with them. In the case of $\mathcal{L}_d$, the associated dual space is labeled as $\mathcal{L}^*_d$. This dual space is completely analogous to the dual space defined for $\mathcal{H}_d$, denoted $\mathcal{H}^*_d$. Just as kets are elements of $\mathcal{H}_d$ and bras are elements of  $\mathcal{H}^*_d$, superkets are elements of $\mathcal{L}_d$ and superbras are elements of $\mathcal{L}^*_d$. With these two spaces, it's possible to define an inner product on $\mathcal{L}_d$

\begin{align*}
    \superbracket{a,b}{a',b'} =& (\bra{a}\otimes\bra{b})(\ket{a'}\otimes\ket{b'}),\\
    =& \braket{a}{a'}\cdot\braket{b}{b'},\\
    =& \delta_{a,a'} \delta_{b,b'}.
\end{align*} Here the mixed product rule for the Kroenecker product has been used and it's assumed that the kets involved belong to an orthonormal basis of $\mathcal{H}_d$. If the inner product of $\mathrm{O}_d$ is the Hilbert-Schmidt inner product, then the transformation performed by $\Upsilon$ is an isomorphism between $\mathcal{O}_d$ and $\mathcal{L}_d$, so the results of products are conserved. The aim of this paper is to teach a projection technique used to adiabatically eliminate certain energy levels in the time evolution of quantum systems. The time evolution of these systems is modeled via the Liouville equation. In Liouville space, the Liouville equation for a density operator $\rho$ whose time evolution is dictated by a Hamiltonian $H$ is given by

\begin{equation}\label{eq:LiouvilleSpaceLiouvillian}
     \dot{\rho} = -\frac{i}{\hbar}(H\otimes \mathcal{I}_d - \mathcal{I}_d \otimes H^T )\rho = \mathrm{L}\rho.
\end{equation} It is important to note that the identity is the identity for the operator space. $\mathrm{L}$ is a super-operator that dictates the time evolution of the density operator in Liouville space. It is also useful to define the identity super-operator, which is an element of $\mathcal{S}_{d^2}$ and fulfills

\begin{equation}
    \mathcal{I}_{d^2}\superket{a,b} = \superket{a,b}.
\end{equation}

The identity is simply

\begin{equation}
    \mathcal{I}_{d^2}= \sum_{a=1}^d \sum_{a'=1}^d \supermatrix{a,a'}{a,a'}.
\end{equation} 

\section{Three level atom}\label{sect:ThreeLevelAtom}

The first example consists of a three level atom in the $\Lambda$ configuration, such as the one in \cite{Ying3LevelAtom1996}. This system consists of an atom where only 3 different energy levels are considered. Two of these levels, labeled $a$ and $b$, are close to each other and the third level, labeled $e$, is farther way from the other two. The atom interacts with two separate lasers, one with frequency $\omega_a$ which couples the $a$ and $e$ levels with detuning $\lambda_a$ and one with frequency $\omega_b$ which couples the $b$ and $e$ levels with detuning $\delta_b$. There is no direct coupling between the $a$ and $b$ levels. In the case where the $e$ level is highly detuned, the objective is to eliminate it adiabatically, following a procedure similar to the one found in \cite{BrionAdiabatic2006}, but without the use of Green's functions. Following that paper's notation,  the system's Hamiltonian in the rotating wave approximation is

\begin{equation}\label{eq:Lambda_Hamiltonian}
    \widehat{H}=\hbar\left[\begin{array}{ccc}
-\frac{\delta}{2} & 0 & \frac{\overline{\Omega_{a}}}{2} \\
0 & \frac{\delta}{2} & \frac{\overline{\Omega_{b}}}{2} \\
\frac{\Omega_{a}}{2} & \frac{\Omega_{b}}{2} & \Delta
\end{array}\right].
\end{equation} Here $\Delta = \frac{\delta_a+\delta_b}{2}$ and $\delta = \delta_a-\delta_b$. $\Omega_a$ and $\Omega_b$ are the Rabbi frequencies of the two lasers. The zero for the energy spectrum is chosen to be the midpoint between the $a$ and $b$ levels. The lasers are assumed to be operating far from resonance such that $\abs{\Delta} \gg \abs{\delta},\abs{\Omega_a}, \abs{\Omega_b}$.The initial condition for the system is assumed to include only the two bottom levels. Since the third level is far detuned from the coupling lasers the objective is to adiabatically eliminate it and find an effective two level Hamiltonian for the system where the $e$ state serves only as a virtual state\cite{Gerry3LA1990}. The projection technique employed here is quite similar to those employed in \cite{NakajimaProyectors1958} and \cite{ZwanzigProyectors1960}. The core idea is to split the eigenvalue space into two parts. One part is the "relevant" part and the other the "irrelevant" part. By projecting the system's Liouville equation into both spaces and finding a closed equation for the relevant part of the system, an effective evolution equation for only the degrees of freedom that are of interest can be found. Physically, the procedure is justified by choosing relevant and irrelevant parts in a way such that they correspond to different time scales.

\subsection{Projection of the Lambda Atom}

For the purpose of this section, the discussion will be limited to the three level atom described in section \ref{sect:ThreeLevelAtom}. It is convenient to employ superkets formed from the $\ket{a},\ket{b},\ket{e}$ basis of $\mathcal{H}_3$ as a basis for $\mathcal{L}_3$. That is all nine superkets of the form $\superket{i,j}$ with $i, j \in [a,b,e]$. The projection technique that will be employed to adiabatically eliminate the third energy level requires that the differential equation that describes the system's time evolution be projected into two different eigenvalue sub-spaces. These two sub-spaces are typically chosen to correspond to different physical timescales for the processes involved, a fast evolution space and a slow evolution space. The projection super-operators must fulfill certain properties. As a super-operator, it must map super-kets into super-kets. If $P$ is a projection super-operator, then $P^2 = P$. The slow and fast spaces together must account for the entirety of the space, so if $P$ projects into the slow evolution space and $Q$ projects into the fast evolution space, then

\begin{equation}\label{eq:ProyectorPropertyIdentity}
    P+Q = \mathcal{I}_{d^2}.
\end{equation} Another property, which is inherent to projection super-operators is

\begin{equation}
    P^2 = P \quad \& \quad Q^2 = Q.
\end{equation} In the superket formalism, these super-operators can be expressed as matrices. The three level atom Hamiltonian acts on vectors which belong to a Hilbert space of dimension $d = 3$. The Hamiltonian is then an operator which belongs to a Hilbert space of dimension $d^2 = 9$. In the superket formalism, density operators, which belong to the same Hilbert space as the Hamiltonian, are expressed as $d \times 1$ vectors (with $d=9$ for the case of the three level atom) and super-operators are expressed as $d \times d$ matrices, so they belong to a space of dimension $d^2 \times d^2 = 81$. The superket basis that will be employed has two indices, this can be re-labeled to only have one index, but the current system will be maintained as it makes the connection to the physics more intuitive. The indices run as follows, for example in the case of the identity

\begin{center}
\begin{blockarray}{cccccccccc}
$\superket{a,a}$ & $\superket{a,b}$ & $\superket{a,e}$ & $\superket{b,a}$ & $\superket{b,b}$ & $\superket{b,e}$ & $\superket{e,a}$ & $\superket{e,b}$ & $\superket{e,e}$ \\
\begin{block}{(ccccccccc)c}
  1 & 0 & 0 & 0 & 0 & 0 & 0 & 0 & 0 & $\superbra{a,a}$\\
  0 & 1 & 0 & 0 & 0 & 0 & 0 & 0 & 0 & $\superbra{a,b}$\\
  0 & 0 & 1 & 0 & 0 & 0 & 0 & 0 & 0 & $\superbra{a,e}$\\
  0 & 0 & 0 & 1 & 0 & 0 & 0 & 0 & 0 & $\superbra{b,a}$\\
  0 & 0 & 0 & 0 & 1 & 0 & 0 & 0 & 0 & $\superbra{b,b}$\\
  0 & 0 & 0 & 0 & 0 & 1 & 0 & 0 & 0 & $\superbra{b,e}$\\
  0 & 0 & 0 & 0 & 0 & 0 & 1 & 0 & 0 & $\superbra{e,a}$\\
  0 & 0 & 0 & 0 & 0 & 0 & 0 & 1 & 0 & $\superbra{e,b}$\\
  0 & 0 & 0 & 0 & 0 & 0 & 0 & 0 & 1 & $\superbra{e,e}$\\
\end{block}
\end{blockarray}
\end{center}
The proposed projection super-operators are
\begin{equation}
    P = \begin{pmatrix}
    1&0&0&0&0&0&0&0&0 \\
    0&1&0&0&0&0&0&0&0 \\
    0&0&0&0&0&0&0&0&0 \\
    0&0&0&1&0&0&0&0&0 \\
    0&0&0&0&1&0&0&0&0 \\
    0&0&0&0&0&0&0&0&0 \\
    0&0&0&0&0&0&0&0&0 \\
    0&0&0&0&0&0&0&0&0 \\
    0&0&0&0&0&0&0&0&1 
    \end{pmatrix},
\end{equation}

and

\begin{equation}
    Q = \begin{pmatrix}
    0&0&0&0&0&0&0&0&0 \\
    0&0&0&0&0&0&0&0&0 \\
    0&0&1&0&0&0&0&0&0 \\
    0&0&0&0&0&0&0&0&0 \\
    0&0&0&0&0&0&0&0&0 \\
    0&0&0&0&0&1&0&0&0 \\
    0&0&0&0&0&0&1&0&0 \\
    0&0&0&0&0&0&0&1&0 \\
    0&0&0&0&0&0&0&0&0 
    \end{pmatrix}.
\end{equation} which can be easily seen to fulfill equation \eqref{eq:ProyectorPropertyIdentity}. In Liouville space, the Liouville equation is \eqref{eq:LiouvilleSpaceLiouvillian}, so the Liouville super-operator can be easily calculated from the Hamiltonian \eqref{eq:Lambda_Hamiltonian}. It is convenient to split the Liouville super-operator as

\begin{equation}\label{eq:ThreeLevelAtomLiouvillian}
    \mathrm{L} = \mathrm{L}_0 + \mathrm{L}_1,
\end{equation}

with

\begin{equation} \label{}
   \mathrm{L}_0 = i\begin{pmatrix}0 & 0 & 0 & 0 & 0 & 0 & 0 & 0 & 0\\
0 & \delta  & 0 & 0 & 0 & 0 & 0 & 0 & 0\\
0 & 0 & \Delta+\frac{ \delta }{2}& 0 & 0 & 0 & 0 & 0 & 0\\
0 & 0 & 0 &  -\delta  & 0 & 0 & 0 & 0 & 0\\
0 & 0 & 0 & 0 & 0 & 0 & 0 & 0 & 0\\
0 & 0 & 0 & 0 & 0 & \Delta-\frac{ \delta }{2}  & 0 & 0 & 0\\
0 & 0 & 0 & 0 & 0 & 0 & -\Delta-\frac{ \delta }{2}  & 0 & 0\\
0 & 0 & 0 & 0 & 0 & 0 & 0 &  -\Delta +\frac{ \delta }{2} & 0\\
0 & 0 & 0 & 0 & 0 & 0 & 0 & 0 & 0\end{pmatrix},
\end{equation} and

\begin{equation}\mathrm{L}_1 = i\begin{pmatrix}0 & 0 & \frac{ \overline{\Omega_a}{2}} & 0 & 0 & 0 & -\frac{  \overline{\Omega_a}}{2} & 0 & 0\\
0 & 0 & \frac{  \overline{\Omega_b}}{2} & 0 & 0 & 0 & 0 & -\frac{  \overline{\Omega_a}}{2} & 0\\
\frac{  \Omega_a}{2} & \frac{  \Omega_b}{2} & 0 & 0 & 0 & 0 & 0 & 0 & -\frac{  \overline{\Omega_a}}{2}\\
0 & 0 & 0 & 0 & 0 & \frac{  \overline{\Omega_a}}{2} & -\frac{  \overline{\Omega_b}}{2} & 0 & 0\\
0 & 0 & 0 & 0 & 0 & \frac{  \overline{\Omega_b}}{2} & 0 & -\frac{  \overline{\Omega_b}}{2} & 0\\
0 & 0 & 0 & \frac{  \Omega_a{2}} & \frac{  \Omega_b}{2} & 0 & 0 & 0 & -\frac{  \overline{\Omega_b}}{2}\\
-\frac{  \Omega_a}{2} & 0 & 0 & -\frac{  \Omega_b}{2} & 0 & 0 & 0 & 0 & \frac{ i \overline{\Omega_a}}{2}\\
0 & -\frac{  \Omega_a}{2} & 0 & 0 & -\frac{  \Omega_b}{2} & 0 & 0 & 0 & \frac{  \overline{\Omega_b}}{2}\\
0 & 0 & -\frac{  \Omega_a}{2} & 0 & 0 & -\frac{  \Omega_b}{2} & \frac{  \Omega_a}{2} & \frac{  \Omega_b}{2} & 0\end{pmatrix}.\end{equation} The property
\begin{equation}
    P\mathrm{L}_1P = 0,
\end{equation} is also fulfilled.  The  property means that $\mathrm{L}_1$ does not map states in the $P$ space into states in the $P$ space. The particular choices for $P$ and $\mathrm{L}_0$ mean that the $P$ space contains all of the $\mathrm{L}_0$ eigenvalues that do not contain the parameter $\Delta$. As this parameter is the largest parameter, the eigenvalues that depend on it will oscillate more rapidly than the rest. The projection technique begins with the Liouville equation

\begin{equation}
    \dot{\rho} = \mathrm{L}\rho,
\end{equation} with $\mathrm{L}$ the one in equation \eqref{eq:ThreeLevelAtomLiouvillian}. However, it is convenient to perform all calculations in the so-called dissipation picture. In this picture

\begin{align}
    \overline{\mathrm{L}}_1 =& e^{-\mathrm{L}_0 t}\mathrm{L}_1e^{\mathrm{L}_0 t},\\
    \overline{\rho}=&e^{-\mathrm{L}_0 t}\rho.
\end{align} In this picture, the Liouville equation is

\begin{equation}
    \overline{\dot{\rho}} = \overline{\mathrm{L}}_1 \overline{\rho}. 
\end{equation} The Liouville equation is projected into both the $P$ and $Q$ spaces

\begin{align}
    P\dot{\rho} =& P\overline{\mathrm{L}}_1\rho,\\
    Q\dot{\rho} =& Q\overline{\mathrm{L}}_1\rho.
\end{align} Equation \eqref{eq:ProyectorPropertyIdentity} is then employed in each of the two equations

\begin{align}
    P\dot{\rho} =& P\overline{\mathrm{L}}_1(P+Q)\rho,\\
    Q\dot{\rho} =& Q\overline{\mathrm{L}}_1(P+Q)\rho,  
\end{align} leading to

\begin{align}
    P\dot{\rho} =& P\overline{\mathrm{L}}_1P\rho+P\overline{\mathrm{L}}_1Q\rho,\\
    Q\dot{\rho} =& Q\overline{\mathrm{L}}_1P\rho+Q\overline{\mathrm{L}}_1Q\rho.
\end{align} The equation for the $Q$ space is then solved via formal integration resulting in

\begin{equation}
    Q\overline{\rho} = \cancel{Q\overline{\rho_0}} + \int_{t_0}^t Q\overline{\mathrm{L}}_1P\overline{\rho}(t') dt'+\cancel{\int_{t_0}^t Q\overline{\mathrm{L}}_1Q\overline{\rho}(t') dt'}.
\end{equation} the initial condition $\overline{\rho}_0$ can be chosen so as to have no part in $Q$ which means the first term on the right hand side can be eliminated. As can be verified by straightforward matrix multiplication using the definitions of the projector and of $\mathrm{L}_1$, not only is $P\mathrm{L}_1 P$, $Q\mathrm{L}_1Q$ is also identically zero. This leads to the elimination of the third term. $\overline{\rho}$ is usually approximated as the initial condition $\overline{\rho}_0$ within the remaining integral. This approximation bears an explanation in further detail. It is assumed that the initial condition evolves according to $\mathrm{L}_0$ so that

\begin{equation}\label{eq:InitialConditionDef}
\rho(t) = e^{\mathrm{L}_0 t}\rho(t_0),
\end{equation} and that $\rho(t_0)$ is completely in $P$ so that it can be expressed as

\begin{equation}\label{eq:InitialConditionDefPSpace}
    \rho(t_0) = \sum_{\lambda_p} \rho_{\lambda_p}(t_0),
\end{equation} where the sum is over all of the sub-states in $P$. Returning to the term $\int_{t_0}^t Q\overline{\mathrm{L}}_1P\overline{\rho}(t') dt'$ and using the definition of the dissipation picture, this term can be written as

\begin{align}
    \int_{t_0}^t Q\overline{\mathrm{L}}_1P\overline{\rho}(t') dt'=&\int_{t_0}^t Qe^{-\mathrm{L}_0 t'}\mathrm{L}_1e^{\mathrm{L}_0 t'}Pe^{-\mathrm{L}_0 t'}\rho(t') dt', \nonumber \\
    =&\int_{t_0}^t Qe^{-\lambda_q t'}\mathrm{L}_1e^{\lambda_p t'}Pe^{-\lambda_p t'}\rho(t') dt'.
\end{align} And using equations \eqref{eq:InitialConditionDef} and \eqref{eq:InitialConditionDefPSpace}

\begin{align}
  \int_{t_0}^t Q\overline{\mathrm{L}}_1P\overline{\rho}(t') dt'=&\sum_{\lambda}\int_{t_0}^t q_{\lambda_q} e^{-\lambda_q t'}\mathrm{L}_1e^{\lambda_p t'}p_{\lambda_p} e^{-\lambda_p t'}e^{\mathrm{L}_0 t'}\rho_{\lambda_p}(t_0)dt' \\
  =& \sum_{\lambda}\int_{t_0}^t q_{\lambda_q} e^{-\lambda_q t'}\mathrm{L}_1p_{\lambda_q} e^{\lambda_p t'}\rho_{\lambda_p(t_0)} \nonumber dt' \\
  =& \sum_{\lambda} q_{\lambda_q} \mathrm{L}_1p_{\lambda_p}\rho_{\lambda_p(t_0)} \int_{t_0}^t e^{-\lambda_q t'} e^{\lambda_p t'} dt' \nonumber \\
  =&\sum_{\lambda} q_{\lambda_q} \mathrm{L}_1p_{\lambda_p}\rho_{\lambda_p(t_0)} \int_{t_0}^t e^{(\lambda_p-\lambda_q) t'}dt' \nonumber \\
  =&\sum_{\lambda} q_{\lambda_q} \mathrm{L}_1p_{\lambda_p}\rho_{\lambda_p(t_0)} \Bigg(\frac{e^{(\lambda_p-\lambda_q) t}-e^{(\lambda_p-\lambda_q) t_0}}{(\lambda_p-\lambda_q)}\Bigg).\nonumber
\end{align} This same procedure can be performed for the approximate term where $\overline{\rho(t')}$ is replaced by $\overline{\rho(t_0)}$

\begin{align}
  \int_{t_0}^t Q\overline{\mathrm{L}}_1P\overline{\rho}(t_0) dt'\approx&\sum_{\lambda}\int_{t_0}^t q_{\lambda_q} e^{-\lambda_q t'}\mathrm{L}_1e^{\lambda_p t'}p_{\lambda_p} e^{-\lambda_p t_0}\rho_{\lambda_p}(t_0)dt'\\
  =&\sum_{\lambda}\int_{t_0}^t q_{\lambda_q} e^{-\lambda_q t'}e^{\lambda_p t'}\mathrm{L}_1p_{\lambda_p} e^{-\lambda_p t_0}\rho_{\lambda_p}(t_0)dt' \nonumber\\
  =&\sum_{\lambda} q_{\lambda_q}\mathrm{L}_1p_{\lambda_p} \rho_{\lambda_p}(t_0) \int_{t_0}^te^{\lambda_p-\lambda_q t'}dt'e^{-\lambda_p t_0} \nonumber\\
   =&\sum_{\lambda} q_{\lambda_q}\mathrm{L}_1p_{\lambda_q} \rho_{\lambda_p}(t_0) \Bigg(\frac{e^{(\lambda_p-\lambda_q) t}-e^{(\lambda_p-\lambda_q) t_0}}{\lambda_p-\lambda_q}\Bigg)e^{-\lambda_p t_0}. \nonumber
\end{align} The difference is thus a phase in the case where $\lambda_q \gg \lambda_p$ this difference can be neglected.This approximation is thus implemented and results in

\begin{equation}
    Q\overline{\rho} = \int_{t_0}^t Q\overline{\mathrm{L}}_1P\overline{\rho}_0 dt'.
\end{equation} This equation is substituted in the equation for the $P$ space

\begin{equation}
   P\dot{\overline{\rho}} = (\cancel{P\overline{\mathrm{L}}_1P\rho}+P\overline{\mathrm{L}}_1\int_{t_0}^t Q\overline{\mathrm{L}}_1 P\overline{\rho}_0 dt').  
\end{equation} This is a closed equation for the evolution of the $P$ sub-space part of the density operator. Employing the definition for the decay picture

\begin{equation}
   P\dot{\overline{\rho}} = Pe^{-\mathrm{L}_0 t}\mathrm{L}_1e^{\mathrm{L}_0 t}\int_{t_0}^t Qe^{-\mathrm{L}_0 t}\mathrm{L}_1e^{\mathrm{L}_0 t} Pe^{-\mathrm{L}_0 t_0}\rho_0 dt'.  
\end{equation} Here the fact that $\rho_0 = \rho(t_0)$ has been used. It is often useful to express the projection super-operators as a sum of super-operators that each project into a specific eigenvalue sub-space. That is to say

\begin{align}
    P =& \sum_{\lambda_p} p_{\lambda_p},\\
    Q =& \sum_{\lambda_q} q_{\lambda_q},
\end{align} where each sum is over all possible eigenvalues in the corresponding sub-space. This leads to 

\begin{equation}
   P\dot{\overline{\rho}} = \sum_{\lambda} p_{\lambda_p} e^{-\mathrm{L}_0 t}\mathrm{L}_1e^{\mathrm{L}_0 t}\int_{t_0}^t  q_{\lambda_q}e^{-\mathrm{L}_0 t'}\mathrm{L}_1e^{\mathrm{L}_0 t'}p_{\lambda_p} e^{-\mathrm{L}_0 t_0}\rho_0 dt'.  
\end{equation} Only the $Q$ projector has been expanded as all $P$ space eigenvalues are 0. The $e^{\mathrm{L}_0 t}$ super-operator is included into the integral 

\begin{equation}
   P\dot{\overline{\rho}} = \sum_{\lambda} p_{\lambda_p} e^{-\mathrm{L}_0 t}\mathrm{L}_1\int_{t_0}^t  e^{\mathrm{L}_0 t}q_{\lambda_q}e^{-\mathrm{L}_0 t'}\mathrm{L}_1e^{\mathrm{L}_0 t'}p_{\lambda_p} e^{-\mathrm{L}_0 t_0}\rho_0 dt'.  
\end{equation} 

The $\mathrm{L}$ super-operators can be applied to the projectors 

\begin{align}
    =& \sum_{\lambda} p_{\lambda_p} e^{-\lambda_p t}\mathrm{L}_1\int_{t_0}^t  e^{\lambda_q t}q_{\lambda_q}e^{-\lambda_q t'}\mathrm{L}_1e^{\lambda_p t'}p_{\lambda_p} e^{-\lambda_p t_0}\rho_0 dt' \\
    =& \sum_{\lambda} p_{\lambda_p} \cancel{e^{-\lambda_p t}}\mathrm{L}_1e^{\lambda_q t}\Bigg(\int_{t_0}^te^{-\lambda_q t'+\lambda_p t'}dt'\Bigg)e^{-\lambda_p t_0}  q_{\lambda_q}\mathrm{L}_1p_{\lambda_p} \rho_0.\label{eq:grouped_integral_3LA}
\end{align} Here the terms have been re-arranged taking advantage that the exponentials, once applied to the projectors, are simply numbers. The first exponential in the RHS of equation \eqref{eq:grouped_integral_3LA} cancels out the same exponential that is on the LHS of the same equation. Before performing the integration, the variable change $\Delta t = t-t_0$ is used, leading to

\begin{equation}
    P\dot{\rho}=\sum_{\lambda} p_{\lambda_p} \mathrm{L}_1e^{\lambda_q t}\Bigg(\int_{t-\Delta t}^te^{t'(\lambda_p-\lambda_q) }dt'\Bigg)e^{-\lambda_p( t-\Delta t)}  q_{\lambda_q}\mathrm{L}_1p_{\lambda_p} \rho_0.
\end{equation}

Performing the integration leads involves only what is inside the parenthesis

\begin{equation}
  \Bigg(\int_{t-\Delta t}^te^{t'(\lambda_p-\lambda_q) }dt'\Bigg) = \frac{e^{(\lambda_p-\lambda_q)t}-e^{(\lambda_p-\lambda_q)(t-\Delta t)}}{\lambda_p-\lambda_q}.  
\end{equation} Multiplying by the other exponentials in the RHS which were not integrated leads to

\begin{equation}
    P\dot{\rho}=\sum_{\lambda} p_{\lambda_p} \mathrm{L}_1  q_{\lambda_q}\mathrm{L}_1p_{\lambda_p} \rho_0\Bigg(\frac{e^{\lambda_p\Delta t}-e^{\lambda_q \Delta t}}{\lambda_p-\lambda_q}\Bigg).
\end{equation} Since $\Delta t=t-t_0$, and the initial time can be freely chosen, the change from $\Delta t$ to $t$ can be made with no loss of generalization. It is worth remembering sum is over all eigenvalues, these are

\begin{align*}
    \lambda_q &\in [ i(\pm\Delta+\frac{\delta}{2}), i(\pm\Delta - \frac{\delta}{2} )],\\
    \lambda_p &\in [0,0,0, \pm i\delta].
\end{align*} Employing a generic starting condition in the $P$ subspace such that

\begin{equation}
    \rho(0)_P = \sum_{\lambda_p} c_{\lambda_p} \rho_{\lambda_p},
\end{equation} which comprises the states $\superket{1,1}$,$\superket{1,2}$,$\superket{2,1}$,$\superket{2,2}$ and$\superket{3,3}$, leads to and evolution equation for only the $P$ subspace part of the system. This is a vector equation with five non-zero entries which lead to five uncoupled equations for the time evolution of all the $P$ subspace states. The first entry is

\begin{equation}
    \dot{\rho_P}_1 = -\frac{\ i c_{11} |\Omega_a|^2 \left( 1-{{\ e}^{\ i t\, \left( \Delta+\frac{\delta }{2} \right) }}\right) }{4 \left(\Delta+\frac{\delta }{2} \right) }+\frac{\ i c_{11} |\Omega_a|^2 \left( 1-{{\ e}^{ -i t \left( \Delta+\frac{\delta }{2} \right) }}\right) }{4 \left(\Delta+ \frac{\delta }{2} \right) }.
\end{equation} This entry corresponds to the state $\superket{1,1}$. As can be readily checked, the constant terms cancel out and this first entry simplifies to

\begin{equation}
    \dot{\rho_P}_1 = -\frac{c_{11}|\Omega_a|^2}{2(\Delta + \frac{\delta}{2})}\sin{\big( t(\Delta+\frac{\delta}{2})\big)}.
\end{equation} The equation corresponding to the state $\superket{1,2}$ has a slightly different structure

\begin{equation}
   \dot{\rho_P}_2 =\frac{i{c_{\ensuremath{\mathrm{12}}}}}{4}\Bigg(\frac{ |\Omega_b|^2 \left( {{ e}^{ i t \delta }}-{{ e}^{ i t\, \left( \Delta+\frac{\delta }{2} \right) }}\right) }{ \left( -\Delta + \frac{\delta}{2}  \right)  }+\frac{|\Omega_a|^2\left( {{ e}^{ i t \delta }}-{{ e}^{ i t\, \left(-\Delta+ \frac{\delta }{2} \right) }}\right) }{ \left( \Delta + \frac{\delta}{2}\right) }\Bigg).
\end{equation} this terms is not easily simplified. The term for 


\section{Example optomechanical system}\label{sec:ExampleSystem}

Arguably the simplest and perhaps most physically intuitive implementation of an optomechanical system is a Fabry-Perot cavity with a movable end mirror. These types of systems, comprised by an optical cavity composed of two parallel, semi-transparent mirrors with one of the mirrors being suspended or otherwise able to move under the constraint of a harmonic potential, date back to experiments in gravitational interferometry \cite{AbramoviciLIGO1992}. For the purposes of this exercise,  the standard optomechanical Hamiltonian detailed in \cite{LawMovingMirror1995}

\begin{equation}\label{eq:basic_hamiltonian}
    H=\hbar \omega a^\dagger a + \hbar\nu b^\dagger b - g_0 a^\dagger a x,
\end{equation}  is used. Here, $a$ and $a^\dagger$ operators correspond to the cavity field which is assumed to have only one mode with frequency $\omega$ while the $b$ and $b^\dagger$ operators correspond to the mechanical oscillator which is also assumed to have only one mode with frequency $\nu$. The parameter $g_0$ is the vacuum optomechanical coupling strength, indicating the strength of the interaction between one photon and one phonon, and $x$ is the mechanical oscillator's displacement from equilibrium.  Usually optomechanical coupling can be considered weak enough that a laser pump with power given by $\Omega$ is employed to boost coupling. The usual Hamiltonian employed in the literature incorporates this driving laser in a frame that rotates with the driving laser's frequency, resulting in

\begin{equation}\label{eq:Optomechanic_hamiltonian}
    H_{opt}=-\hbar \delta a^\dagger a + \hbar\nu b^\dagger b - g_0 a^\dagger a x + \Omega(a^\dagger+a),
\end{equation} with $\delta = \omega_L-\omega$. The Hamiltonian \eqref{eq:Optomechanic_hamiltonian} models the evolution of the cavity when there is no dissipation. However, optomechanical systems often include dissipation, for example photon leakage due to the semi-transparent end mirrors, which requires additional considerations.

\subsection{Optomechanical master equation}

Dissipation can be modeled through the use of a master equation \cite{CarmichaelQuantumOptics1999}. A master equation models the interaction between a system of interest, often referred to as the central system, and a larger system often referred to as a reservoir, which represents the environment. In this manner, dissipation is modeled as an energy exchange between the system of interest and the reservoir. Both systems, when combined, form a closed system. During the derivation, the reservoir degrees of freedom are traced out and the result is an equation that only models the evolution of the central system but which accounts for the energy it exchanges with its environment. In the case of the basic optomechanical system introduced above, the master equation is

\begin{equation} \label{eq:OptoMechanicalMasterEquation}
\dot{\rho} = \frac{1}{i\hbar}[H,\rho] +L_a\rho + L_b \rho.
\end{equation} The $\mathrm{L}$ terms are in what is known as Lindblad form 

\begin{align}
L_a \rho =& - \frac{\kappa}{2}(n_p + 1)[a^\dagger a\rho + \rho a^\dagger a -2a\rho a^\dagger]  \\
 &- \frac{\kappa}{2}(n_p)[ aa^\dagger\rho + \rho  aa^\dagger -2a^\dagger\rho a]\, ,\nonumber
\end{align} and

\begin{align}
L_b \rho =& - \frac{\gamma}{2}(n_m + 1)[b^\dagger b\rho + \rho b^\dagger b -2b\rho b^\dagger]  \\
 &- \frac{\gamma}{2}(n_m)[ bb^\dagger\rho + \rho  bb^\dagger -2b^\dagger\rho b]\, .\nonumber
\end{align} $\mathrm{L}_a$ models the dissipation of the electromagnetic field within the cavity and $\mathrm{L}_b$ models the dissipation of the mechanic excitations in the mechanical resonator, there are two dissipation terms as the system is coupled to two different reservoirs, a photonic reservoir and a phononic reservoir. $\kappa$ and $\gamma$ are the respective dissipation coefficients. $n_p$ and $n_m$ are the thermal occupation numbers for the photon reservoir and the mechanical reservoir at the respective resonant frequencies of the modes being considered. In order to find a master equation that models optomechanical cooling it is often necessary to find a convenient eigenbasis for these kinds of super operators, as the $\mathrm{L}$ terms are known. The basis employed here is known as the damping basis and will now be introduced in detail. 

\section{The damping basis}

When working with equations such as \ref{eq:OptoMechanicalMasterEquation}, it is often desirable to be able to write the density operator in a convenient manner.  In the presence of dissipation however, the problem is not necessarily straightforward. Here the use of a basis particularly suited for these types of problems, known as the damping basis \cite{BriegelDampingBasis1993}, is illustrated. For the purpose of this exercise, the discussion is limited to the case of a cavity with photon dissipation, modeled by

\begin{equation}\label{eq:DampingBasisMasterEquation}
    \dot{\rho} = \frac{1}{i\hbar}[\hbar \omega a^\dagger a,\rho] +\mathrm{L}_a\rho = \mathcal{L}\rho.
\end{equation} $\mathcal{L}$ is the complete Liouville super operator which includes both $H$ and $\mathrm{L}$. $\mathcal{L}$ and $\mathrm{L}$ act on the density operator $\rho$ which itself acts on Hilbert space vectors, so sometimes $\mathcal{L}$ and $\mathrm{L}$ are referred to as super operators. If it is assumed that \ref{eq:DampingBasisMasterEquation} has been solved and all of the eigenvalues are known, then it would be possible to write the density operator's initial state as an expansion

\begin{equation}
    \rho(0) = \sum_{\lambda} \check{c}_{\lambda} \hat{\rho}_{\lambda}, 
\end{equation} so that the state of $\rho$ at any future time can be written as

\begin{equation}
    \rho(t) = \sum_{\lambda} \check{c}_{\lambda} e^{\lambda t}\hat{\rho}_{\lambda}. 
\end{equation} In both cases $\lambda$ runs over all possible eigenvalues and accounts for different eigenstates should there be degeneracy. The problem then becomes finding an expression for the coefficients $\check{c}_{\lambda}$. This is done via a trace product

\begin{equation}
    \check{c}_\lambda = Tr[\rho(0)\check{\rho}_\lambda],
\end{equation} where the states indicated with a check are known as "dual" states, which fulfill

\begin{equation}
    Tr[\hat{\rho}_{\lambda} \check{\rho}_{\lambda'}] = \delta_{\lambda,\lambda'}. 
\end{equation} Dual and regular states are different, but correspond to the same eigenvalues

\begin{align}
    \mathcal{L}\hat{\rho}_\lambda =& \lambda \hat{\rho}_\lambda, \\
    \check{\rho}_\lambda\mathcal{L} =& \lambda \check{\rho}_\lambda.
\end{align} The states are given by

\begin{align}
  \hat{\rho}_n^k=& a^{\dagger k} \frac{(-1)^{n}}{(1+n_p)^{k+1}}: L_{n}^{(k)}\left[\frac{a^{\dagger} a}{1+n_p}\right] \exp \left(-\frac{a^{\dagger} a}{1+n_p}\right):  \quad k \geq 0 \nonumber\\
    \hat{\rho}_n^k=& \frac{(-1)^{n}}{(1+n_p)^{|k|+1}}: L_{n}^{(|k|)}\left[\frac{a^{\dagger} a}{1+n_p}\right] \exp \left(-\frac{a^{\dagger} a}{1+n_p}\right): a^{|k|}  \quad k \leq 0 . \nonumber
\end{align} Here the $L_n^k$ are the associated Laguerre polynomials. $k$ and $n$ fulfill

\begin{align}
    n=&0,1,2,\dots \\
    k=&0,\pm 1,\pm 2, \dots
\end{align} and determine the eigenvalues for the dissipative part of $\mathcal{L}$

\begin{equation}
    \lambda_n^k=-\kappa (n+\frac{\abs{k}}{2}).
\end{equation} It's worth noting that the eigenvalues are always non-positive, as these types of states decay over time. The dual states are similar and given by

\begin{align}
    \check{\rho}_n^k=&\left(\frac{-n_p}{1+n_p}\right)^{n} \frac{n !}{(n+k) !}: L_{n}^{(k)}\left[\frac{a^{\dagger} a}{n_p}\right]: a^{k} \quad  k \geq 0, \nonumber \\
   \check{\rho}_n^k=&\left(\frac{-n_p}{1+n_p}\right)^{n} \frac{n !}{(n+|k|) !} a^{\dagger|k|}: L_{n}^{(|k|)}\left[\frac{a^{\dagger} a}{n_p}\right]: \quad k \leq 0, \nonumber
\end{align} with the eigenvalues being the same as for the regular states. It's worth noting that the parameter $n_p$ appears in the eigenstates but not the eigenvalues. In the particular case where the cavity can be considered to be in contact with a bath at zero temperature, the states are given by far simpler expressions

\begin{align}
    \hat{\rho}_n^k=&\nonumber a^{\dagger k}(-1)^{a^{\dagger} a+n}\left(\begin{array}{c}
n+k \\
a^{\dagger} a+k
\end{array}\right) \quad k \geq 0, \\
\hat{\rho}_n^k=& \nonumber (-1)^{a^{\dagger} a+n}\left(\begin{array}{c}
n+|k| \\
a^{\dagger} a+|k|
\end{array}\right) a^{|k|} \quad k \leq 0,
\end{align} for the regular states and 

\begin{align}
    \check{\rho}_n^k =&\nonumber  \frac{n !}{(n+k) !}\left[\begin{array}{c}
a^{\dagger} a \\
n
\end{array}\right] a^{k}=a^{\dagger n} a^{n+k} /(n+k) ! \quad k \geq 0,\\
\check{\rho}_n^k =&\nonumber a^{\dagger|k|}\left(\begin{array}{c}
a^{\dagger} a \\
n
\end{array}\right) \frac{n !}{(n+|k|) !}=a^{\dagger n+|k|} a^{n} /(n+|k|) ! \quad k \leq 0,
\end{align} for the dual states. With the particular forms of these eigenstates, we now proceed to obtain a projected master equation for optomechanical cooling.


\section{Projection of the master equation}

Reiterating the core idea of the method, it is to project the master equation into two different eigenvalue sub-spaces with no overlap between them. One subspace is relevant to the problem we wish to solve and the other one is not. In this particular case, in order to model a cooling process the aim is to split the eigenvalue space into a sub-space corresponding only to states that do not decay quickly in time and one space with states that do, such that at the end of the cooling process there will be only non-decaying states represented. Physically this is justifiable by considering different time scales. In the system considered in section \ref{sec:ExampleSystem}, three different time scales can be considered. This can be illustrated with some experimental parameters.

\begin{table}[]
    \centering
    \begin{tabular}{|c|c|c|c|c|}
        \hline
        Source & $\frac{\nu_m}{2\pi}$ & $\frac{\kappa}{2\pi}$ & $\frac{g_0}{2\pi}$&  $\frac{\gamma}{2\pi}$  \\
        \hline\hline
        Vezio et al \cite{VezioOMExperiment2020} & 530 kHz & 1.9 MHz &  30 Hz & 0.08 Hz \\
        \hline
        Nielsen et al \cite{NielsenMultimodeOptomechanicalMembrane2017} & $\approx$ 1MHz & 14 MHz & 115 Hz & 170 mHz\\
        \hline
    \end{tabular}
    \caption{Caption}
    \label{tab:my_label}
\end{table}

The projection procedure is generally the same as in the case of the 3 level atom \ref{sect:ThreeLevelAtom}. The projection super operators have the same properties, but now encompass different eigenvalue subspaces. The projection super operators are not expressed in super-ket notation in this instance as the eigenvalue spaces are not finite so it is not feasible to employ explicit matrix multiplication. The super operators are written as in \cite{YanesOptomechanicalCooling2020}

\begin{eqnarray}
  P &=& \sum_{\lambda} (\hat{\rho}_{\lambda}^{cav}\otimes\hat{\rho}_{\lambda}^{mec})\otimes(\check{\rho}_{\lambda}^{cav}\otimes\check{\rho}_{\lambda}^{mec}),\label{eq:projector_p}\\
  &=& \sum_\lambda \mathcal{P}_\lambda, \nonumber\\
  Q &=& \sum_{\lambda'} (\hat{\rho}_{\lambda'}^{cav}\otimes \hat{\rho}_{\lambda'}^{mec})\otimes(\check{\rho}_{\lambda'}^{cav}\otimes\check{\rho}_{\lambda'}^{mec})\label{eq:projector_q}\, ,\\
  &=& \sum_\lambda \mathcal{Q}_{\lambda'}, \nonumber
\end{eqnarray}

\section{Adiabatic approximation}

\bibliographystyle{unsrt}
\bibliography{DidacticPaper}
\end{document}
